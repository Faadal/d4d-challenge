\newpage

\section{Methodology}

Let's describe how we have faced D4D, enumerating the different phases of our project and highlighting the corresponding milestones. We really think that explaining how this work was carried out can be useful both to better illustrate our conclusions and results, and to give ideas for similar projects.
\\
\\
First of all, once we had clearly understood the D4D bases, we studied all provided datasets to be certain of what kind of data were available. Next, we began with the research work: getting information of Ivory Coast, studying some papers about behavioural patterns obtained from mobile phones traces, looking for new related datasets...
\\
\\
With this knowledge, we were prepared to decide which lines of work would be more interesting (without forgetting the cooperative and development goal apart from the scientific one) and, what's more important, being aware of our own time constraints and our team skills --being realistic is crucial.
\\
\\
After some discussion, we agreed to focus on the 2nd dataset 'Individual Trajectories: High Spatial Resolution Data (SET2)' \citep{DBLP:journals/corr/abs-1210-0137} , since it seemed to be the most adequate one for our approach. We conducted our analysis according to the following stages:
\\
\\
1) Processing all traces, grouping them by user and sorting them chronologically, as hourly time series. Paying attention to imprecise or weird traces, which must be filtered.
\\
\\
2) Calculating different magnitudes (absolute, relative and normalized ones) and their mean, median and dispersion to display visual charts, which helped us to discover correlations and to identify 'Temporal Commuting Patterns' for each week day~\ref{sec:results}.
\\
\\
3) Handling antenna locations from the previous processed traces, allowed us to identify 'Geospatial Commuting Patterns'. Firstly, we represent networks graphs and some static maps (snapshots of commuters motion). Later, we were able to create animated and detailed maps (Kernel Density, Grids...) which made easier to see crowded areas, related highways... during the days and all across Ivory Coast (Figures~\ref{kde_first_peak} (meter la segunda imagen))
\\
\\
4) Eventually, an online and interactive web-based animation was developed. This geovisualization technique is advantageous in that neither specialized GIS knowledge nor software is required, and it enables change over time visualization that would be difficult to see with static or paper maps. The interface combines raster maps produced in the ArcGIS environment and vector data [PANTALLAZO de la APP final en web]. User interaction is facilitated through the inclusion of buttons on the interface (play controls, modal tab, zooming and panning).
\\
\\
\\
As can be seen, the whole process to obtain the results has been carried out step by step. We had a planning which was useful, but the really important thing was the fact of planning, not the planning itself. There will be unexpected events which required the team to adapt itself to new circumstances.
\\
\\
In the end, we would like to remark how, although assigning particular tasks to different team members looking for productivity, all of us have tried to be involved in all areas.
\\
\\