\newpage

\section{Commuted Communities: Commuting Dynamics 4 a Change}

For decades, Africa has been receiving official development assistance, ODA, from the so-called countries of the “North”. There have been several models of cooperation with this continent, from bi-lateral country agreements passing through religious and non-governmental organizations to new models of financing development. None of them is free from dark sides and controversial issues in this “culture of aid” \citep{moyo2009dead}. The African continent doesn’t need that kind of helping agreements which submerge the continent in a dark hole and position it on unequal relations with international financial institutions which pretending good wishes create difficult conditions that keep on enslaving the continent on new ways of Neo-colonialism. African countries are in need of endogenous conditions for creating their path to democracy and their own government empowering. We understand development as a process linked to the capacity of African people to decide about its own future\citep{sen2000desarrollo}. In this sense, our proposed ways of cooperation pass through the vision of an horizontal structure where sharing-knowledge conditions are created in order to allow the empowering in every direction. As the one we introducing on this article, our proposals are based on the new technologies and the way they can help to develop new aspects as far as planning and researching are concerned in any country they are implemented.
\\
\\
The emergence of new technologies and virtual tools in the field of development cooperation had an early stage in the late nineties\citep{de2005global}. In that period, the developed projects were aimed at providing hardware structures and networks to disadvantaged communities or with communication difficulties due to the orography within their territory. Nevertheless, our proposals on this article are based on software tools that have been already experienced in countries such as Ivory Coast where the development conditions are allowing to start testing some of these tools.
\\
\\
GIS based technologies have been sufficiently proved in some sustainable development projects carried out in Africa\footnote{http://www.esri.com/library/brochures/pdfs/gis-for-africa.pdf} related to international cooperation as well as in local development where the cross-sectorial work is critical and the shortage of funding make it, sometimes, impossible \citep{mitchell1997zeroing}, \citep{craig2002community}. Recently the GIS has a crucial role in the democratic processes in Africa, as we can see in the creation of the Census 2012 in Rwanda where the GPS data collection will enable them to build a comprehensive GIS database, which shows boundaries, location of schools, hospitals and markets covering 17,700 villages\footnote{http://www.humanipo.com/news/1279/GIS-based-enumeration-kicks-off-in-Rwandas-2012-census}.
\\
\\
Also in South Africa, part of our team had the experience of working with GIS as a tool to identify agricultural and rural areas where to start community vegetable gardens as well as managing the cattle in some projects implemented in KwaZulu-Natal region.
\\
\\
As we said, the two tech experiences gathered on this article look deeply their contribution to serve as tools to facilitate the cooperation in two aspects:
\\
\\
a) It will allow to develop better ways to plan and manage the territory since their GIS based structure will permit us to work with geographical layers: landcover, roads map, railways lines, water sources, etc. Consequently, this information will allow to visualize and reach some conclusions in order to design and formulate better cooperation and local development projects based on the municipalities.
\\
\\
b) This tool will become an useful resource in social research\citep{fielding2008sage} on  deterritorialized and transnational realities\citep{sassen2007sociología} since it brings new possibilities of comparative analysis and the creation of resources ad hoc. 

