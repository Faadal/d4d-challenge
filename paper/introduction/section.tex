\newpage
\setcounter{secnumdepth}{1}
\section{Introduction}
Our final aim is detecting spatio-temporal patterns in order to obtain an useful knowledge to better manage the country resources. For example, if we could predict the traffic intensity segmented by road, week day and hour, then another secondary roads could be suggested or the budget for the most used ones could be increased.
\\
In this paper, we propose a Space-Temporal Model to achieve this goal. A Space Model, because user interactions with geolocated antennas are analyzed and treated, and a Temporal Model because several time windows are used to group this user dynamics. Later, the mix of this two variables (space and temporal) are consumed and visualized by a Geographic Information System.
\\
\\
Our aim i related to the Commuting concept and could be defined as follow: \emph{Commuting} is regular travel between one's place of residence and place of work or full-time study \citep{wiki:commuting}, but  sometimes its refers to \emph{any regular or often repeated traveling between locations when not work related}. Our first commuting approach is defined like: "Mobility patterns through inferring dynamic users movements grouped by temporal windows". 
\\
A dynamic user is defined like an user changing his antenna location within the studied temporal window (i.e.: each temporal window groups the whole user communication into one specific hour). Among these temporal windows, static users have been removed, i.e.: users that do not change their antennas locations within the temporal range. The justification to remove these users comes to focus our study on users that are moving into this temporal windows and perform micro-displacements.
