\newpage
\setcounter{secnumdepth}{1}
\section{Introduction}
Our aim and final goal in this project is to detect spatio-temporal patterns in order to obtain an usefull knowledge to manage better the country resources. As example, if we can predict the intensity of traffic segemented by road, day of the week and hour, could be suggested anothers secondary roads or invest into improve the most used ones.
\\
\\
Commuting dynamics are defined as follow: \emph{Commuting} is regular travel between one's place of residence and place of work or full-time study \citep{wiki:commuting}, but  sometimes its refers to \emph{any regular or often repeated traveling between locations when not work related}. Our first commuting approach is defined like: "Mobility patterns through infering dynamic users movements grouped by temporal windows". A dynamic user is defined like an user that change his antenna position into the studied temporal window (i.e.: hour). Between this temporal windows static users habe been removed (i.e.: users that does not change its antennas positions into the temporal range).


This example shown the same antennas, however, we have traces of dynamic users across all country connecting with differents antennas.