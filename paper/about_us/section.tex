\newpage

\section{About us and why face this challenge}

Paradigma Labs Research Group\footnote{http://labs.paradigmatecnologico.com/} (PLRG) core values are conducted by one motivation: "To figure out the fuzzy dynamics between Humanity and Technology", providing tools and methods to study, display and understand these dynamics. Therefore, an international challenge whose research subject can be chosen freely as long as it relates to an objective of development improving quality of life for people, quickly held our attention.
\\
\\
GIS laboratory\footnote{http://humanidades.cchs.csic.es/cchs/sig/} at Centre for Humanities and Social Sciences (CCHS), Spanish National Research Council\footnote{http://www.csic.es/} (CSIC) is a multidisciplinary group with a huge experience in Remote Sensing and Geoprocessing, providing a quality support for plenty of researches carried out at CSIC.
\\
\\
Our final aim has been detecting geospatio-temporal patterns in order to obtain an useful knowledge to better manage the country resources. For example, if we could predict the traffic intensity segmented by road, week day and hour, then another secondary roads could be suggested or the budget for the most used ones could be increased. Through mobile communications, a specific user can be tracked along the day, not only by means of 'call comunications' but also thanks to applications running on their handsets: IMS, RSS... The dataset provided by Orange is a sample, and it only uses 'call communications', however, with the whole set of data (i.e.: app and call comunications), we strongly believe more accurate and complete models could be discovered helping to identify new kinds of dynamics.
\\
\\
From our own experience studying and modeling several kinds of Human Dynamics like during the ESF project DynCoopNet\citep{dyncoopnet2012} and while developing a Business Intelligence Tracking Tool on Twitter \citep{labselecciones}, we can claim there are two main exploring perspectives: the Geographical one and Temporal one. We believe a mathematical model related to Human Dynamics must be managed with these two viewpoints. The Temporal component is useful by providing a tool to go backward and forward in order to get a more detailed understanding of the dynamic, not only moving across the timeline, but also creating temporal windows to group events. The Geographical component provides a more high-level understanding related to the human mobility across the space in different levels and relating it to some other spatial features. Mixing both components in a final and single visualization has led our study during the project.
\\
\\
Consequently, in this paper, we propose a Geospatio-Temporal Model. A Geospatial Model, because user interactions with geolocated antennas are analyzed and treated, and a Temporal Model since several time windows are used to group these user dynamics. The combination of these two variables is used and displayed by a GIS. Initially, several results are showed supporting the project main conclusions. However, what's really important is the whole process for handling the data, that is, the code, tools and methodology, which will be available to the researcher comunity, allowing to study more deeply the dynamics. For instance, a Standard Kernel Density estimation (KDE) aims to produce a smooth density surface of spatial point events over a 2-D geographic space\citep{SIM:SIM4780090616,5969036}, final dynamics visualization across the several days of the week will be shown by means of KDE, in order to understand and proof which an where are the maximum commuting peaks. 
\\
\\
We have focused on the Commuting concept, which could be defined as follows: \emph{Commuting} is regular travel between one's place of residence and place of work or full-time study \citep{wiki:commuting}, but  sometimes it refers to \emph{any regular or often repeated traveling between locations when not work related}. Our first commuting approach is defined like: "Mobility patterns through inferring dynamic users movements grouped by temporal windows". 
\\
\\
A {\it commuter} or {\it dynamic user} is defined as an user changing his antenna location within the studied temporal window (i.e.: each temporal window groups the whole user communication during a specific hour). Among these temporal windows, {\it non-commuters} or {\it static users} have been removed, i.e.: users who do not change their antennas locations within the temporal range. The justification to remove these users comes to focus our study on users that are moving into this temporal windows and perform micro-displacements. It is common that a same user performs these two kind of dynamics within the same temporal window. Note that we are not quantifying the distance, but only the fact of changing from a particular antenna to another one.
\\
\\