\newpage

\section{Representation and visualization}

 {\color{red} CSIC, estado del arte de gis, intersección con humanidades, whatever}
\\
 {\color{red} CSIC/PLRG, Poner los enfoques por los que hemos pasado: los puntos de colores por intensaida de paso, la grilla y finalmente, acabar hablando del kernel density como herramienta final de visualización} 
 \\
 \\
To visualize commuting patterns, a diferent approach is used here. To apply a Kernel Density estimation, a geographic network is modeled in order to represent, on the one hand, the antennas positions like nodes on a map and on the other hand, in a time-line order, the user displacements between antennas. With this network, a indegree rank is calculated in order to meassure and set up the weight of each antenna (KDE will use this parameter) and the edge representation is removed from the visualization.
