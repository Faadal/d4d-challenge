\newpage
\section{Conclusions}

After having completed the project, we have demonstrated with a practical example how relevant conclusions can be extracted by analyzing this kind of mobile phone networks data. In particular, a couple of general ideas have been confirmed:
\\
\\
a) Mobile phone traces are all over the world and they hold plenty of high value information.
\\
\\
b) Among possible behavioural patterns extracted from a), knowing people dynamics, and specially commuting patterns, constitutes a valuable tool to improve infrastructure and public services both for the time being (detecting crowded/empty areas or periods) and for the future (predictions).
\\
\\
Focusing on D4D, we could remark these partial findings:
\\
\\
a) Distinction of {\it commutters} and {\it noncommuters} groups and their evolution during each week day and for different Ivory Coast regions.
\\
\\
b) Guessing of common usage of mobile phones, in amount of calls terms, during a day and in different cities.
\\
\\
c) Identification of a morning peak (9:30am) and an evening peak (7:30pm), in maximum displacement terms, values have been normalized using the median of total amount of commuters.
\\
\\
d) Identification of lower valley between both peaks.
\\
\\
As the main conclusion for this project, it could be claimed that, basing on the collected Orange mobile phone traces in Ivory Coast during the observed period of time, {\bf a couple of commuting peaks have been identified for each day of the week, with a more defined pattern for work days and for big cities (Abidjan, Bouaké, Daloa, Yamoussoucro)}. Moreover, {\bf people motion between the outskirts and the city center in the morning, and vice versa in the evening could be detailed for each particular city, highlighting those road segments with more traffic}.
\\
\\
Apart from the 'commuting' conclusions, there is another one which deserves to be exposed. After some discussions, we decided to create an accesible, user-friendly and customizable tool so that this kind of data could be actually profitable. How can you expect this complex process can be understood for some common people if you do not make things easy ?
\\
\\
Summing up, we hope this global D4D effort can help Ivory Coast in decision making for policy measures and ultimately leads to perform some trustworthy plans to improve {\it ivoriens} quality of life. Taking advantage of a tool like this one will save them money and time.
\\
\\