\newpage
\section{Final Conclusions}

Summing up the methods and tools used in this novel study, a mathematical model to shape Space and Temporal user-antenna communications has been proposed, two kind of user dynamics have been identified in order to carry out and focus the study on the commuting dynamics. A commuting pattern model is showed to test with the Orange sample dataset of user communications.
\\
\\
A temporal set of windows defined in 60 minutes ranges and grouped by hours and day of the week has been proposed in the model, however, this assumption could be changed in order to explore the data with another point of view.
\\
\\
Dynamic users has been splitted from statis users to detect commuting patterns, an the mathematical model proposed and appliyed has shown two maximum displacement peaks and a central valley following the main component of total call numbers, normalizing the maximun displacement using the median with the total number if dynamic users.
\\
\\
A visualization of the displacement with a Kernel Density estimation by means of GIS using the antennas positions and users displacements like a direct graph has shown the contraction and the expansion of the commuting dynamics across the 24 hours of the day.  A tool to visualizate paralell this dynamic  across the seven days of the week is release on line\footnote{"Poner la dirección de las herramientas"}.

{\color{red} Poner }



\section{Future work}

We strongly belive that working with more detailed data i.e.: data from applications and other kind of communications, researchers will be able to work in more accurate mathematical models and therefore, more accurate and useful visualizations. GIS tools are itself a method to research on geolocated data. Several algorithms and methods could be applied to figure out new dynamics and to return the profit of these results into a new develop policys in order to improve the human condition.