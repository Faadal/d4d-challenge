\newpage
\section{Recommendations for further work}

This project has reached interesting results, but it can still be evolved in several ways. Due to not having enough time, some ideas were only proposed and they could not be developed. They have been listed here below as an outline for those teams who agreed with us in seeing their potential related to the model.
\\
\\
a) {\bf Particularization}: general findings are useful to know how to face a problem initially, but with concrete findings more adequated solutions will be reached. Specific conclusions and maps for different cities and regions could be calculated.
\\
\\
b) Clustering and/or filtering antennas: by traffic (using the 1st dataset), by location (city/field, latitude)...
\\
\\
c) Replicating the model and results with the 3rd dataset.
\\
\\
d) A 60-minutes time span and a daily displaying approach have been used. However, both assumptions could be modified in order to explore data from another perspective: season patterns, overlapped or shorter time spans...
\\
\\
e) Looking for more understandable charts: normalizing time series by dividing all values by the maximum of the day... and some other strategies to obtain relative magnitudes.
\\
\\
f) Trying to find correlation evidence between the amount of calls rate and prosperity indicators (business, grants...).
\\
\\
g) Looking for additional socio-economic datasets to conduct new mixed analyses (weather...).
\\
\\
h) Developing a new kind of maps based on tessellations (i.e., Voronoid diagrams), which have been proved to be really useful for this kind of studies.
\\
\\
i) Applying DTW\footnote{http://en.wikipedia.org/wiki/Dynamic\_time\_warping} or LCS\footnote{http://en.wikipedia.org/wiki/Longest\_common\_subsequence\_problem} algorithms to discover similarities among different traces series, in order to identify types of areas (residential, commercial, business) and to predict its evolution in mobility terms.
\\
\\
j) Trying to identify where people live and work.
\\
\\
k) {\bf Outliers detection}: both in particular time zones and in specific regions, or also regarding patterns evolution and consolidation.
\\
\\
Moreover, some ideas and modules of this project are hoped to be used into another completely different works, so that original and novel conclusions are found with a high synergy value.
\\
\\
Eventually, we strongly believe that working with more detailed data (both Internet and Call/SMS communications) will let researchers find more accurate mathematical models and therefore, more precise and useful visualizations. Furthermore, having longer time span datasets available would assert the patterns and would  allow to describe more reliable people dynamics trends.
\\
\\
GIS tools are themselves a fantastic way to research on geolocated data. Applying different GIS algorithms and methods could obtain new dynamics patterns, providing profitable new results to be taken into account when better development policies are decided in order to improve the people quality of life.